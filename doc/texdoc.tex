%#!xelatex
% Texdoc user manual
% Copyright 2008-2021 Manuel Pégourié-Gonnard and Takuto Asakura
% distributed under the terms of GPL v3 or later
\documentclass{texdoc-doc}

\title{Texdoc}
\subtitle{Find \& view documentation in \TL}
\pkgurl{https://tug.org/texdoc/}
\author{Manuel Pégourié-Gonnard\and Takuto Asakura}
\date{v3.3\quad \today}

\begin{document}

\VerbatimFootnotes

\maketitle

\section{Quick Guide}

Texdoc is a command-line tool to find and view documents in {\TL}. Typing
%
\begin{htcode}
texdoc «name»
\end{htcode}
%
in your command line, you will see a document of the |«name»| package is popped
up. Of course, you have to replace |«name»| with the actual name of a package.
To look up the documentation of more than one packages at once, just give
multiple |«name»|s as arguments.

\subsection{Modes}
\label{sec:modes}

Texdoc has several modes that determine how results will be returned. The
default is \emph{view} mode, which opens the first, that is supposedly the
best, result with a suitable viewer. It is rather handy when you know what you
want to read. On the other hand, there may be other relevant documents for the
given |«name»|, which are ignored in the view mode.

In the \emph{list} mode, Texdoc lists all relevant documentation and ask you
which one you want to view. This mode is useful when there are other
interesting sources of information in addition to the main documentation of a
package.

There is also a \emph{mixed} mode, intended to get the best of the view mode
and list mode: if there is only one good result, then Texdoc opens it in a
viewer, like in the view mode. Otherwise, it offers you a menu, like in the
list mode.

By default, Texdoc hides some results, which expected to be less relevant,
unless it cannot find any relevant result. In the \emph{showall} mode, Texdoc
always shows all results, including bad ones.

A couple of command-line options are available for selecting the mode to
execute: \sopt{w} (\lopt{view}) for the view mode, \sopt{m} (\lopt{mixed}) for
the mixed mode, \sopt{l} (\lopt{list}) for the list mode, and \sopt{s}
(\lopt{showall}) for the showall mode.

If you have your favorite mode and always use it, you may not want to keep
typing the same option. The next section describes how to customize Texdoc
using its configurations files.

\subsection{Configuration files}
\label{sec:quick-file}

The configuration file enables you to tweak Texdoc in many ways. You can use
the \lopt{files} option to know where to put your personal configuration file;
you may need to create this file, possibly with some parent directories. If you
want to know the full list of possible configuration files, see
Section~\ref{sec:prec}.

To set your favorite mode, just insert a line |\ci{mode} = «mode»| in your
personal configuration file, where |«mode»| is one of |view|, |mixed|, |list|,
and |showall|. Though your preferred language is usually detected automatically
by getting the system locale, you can set it explicitly with a line
|\ci{lang} = «2-letter code»| in the configuration file.

\subsection{Viewers}
\label{sec:viewer}

The way of Texdoc for choosing a viewer varies according to your platform. On
Windows, macOS, or Unix with KDE, GNOME, or XFCE, it uses your file
associations like when you double-click files in the Explorer, the Finder or
your default file manager (except for the text viewer, which is always a
pager). Otherwise, it tries to find a viewer in the path from a list of
``known'' viewers.

You may want to use a different viewer for some types of documents. This can be
achieved by setting the various \ci{viewer\_\meta{ext}} configuration items,
where |«ext»| is an extension corresponding to a file type. For example, if
you want to set \code{xpdf} as your default PDF viewer, and run it in the
background, insert the line |viewer_pdf = xpdf %s &| in your configuration
file. Herein, |%s| is a place holder for the name of the file to view.

\subsection*{\underline{You can stop reading now}}

The following parts explain the mechanisms of Texdoc for finding the best
results and how to customize them. We have been trying hard to optimize the
default configuration values so that normal users do not need to fiddle with
it. Thus, the following parts are useful only when you are curious or have
special needs.

\clearpage

\section{Controlling Texdoc}

\subsection{Command-line options}
\label{sec:cl}

The full usage of Texdoc can be summarized as follows:
%
\begin{htcode}
texdoc «names»
texdoc «options» «names»
texdoc «action»
texdoc «options» «action»
\end{htcode}

We have tried to implement a GNU-compatible option parser. Short options, each of
which consists of a single letter, must start with a single hyphen |-|.
Multiple short options can be specified with a single hyphen, \eg |-vl| is
equivalent to |-v -l|. Long options have to be following double hyphens |--|.
All options must be specified before the first argument. A String beginning
with a hyphen after the first argument will be treated as an argument starting
with a hyphen.

Some options are called action options, which can be used for |«action»| in the
above usage. Four actions are available:

\begin{clopt}{\sopt{h}, \lopt{help}}
Shows a quick help message (namely a list of command-line options) and exit
successfully.
\end{clopt}

\begin{clopt}{\sopt{V}, \lopt{version}}
Shows the current version of the program and exit successfully.
\end{clopt}

\begin{clopt}{\sopt{f}, \lopt{files}}
Shows the list of configuration files for the current installation and
platform, with their status and exit successfully. Normally, only ``active''
and ``disabled'' files are shown (see~\ci{lastfile\_switch}). To show ``not
found'' files as well, you can use \lopt{verbose}.
\end{clopt}

\begin{clopt}{\code{\lopt{just-view} \meta{file}}}
Open |«file»| in the usual viewer. The file should be given with full path,
absolutely no searching is done. This option is not really meant for users,
but rather intended to be used from another program, like a GUI front-end to
Texdoc.
\end{clopt}

Some normal options such as \sopt{v} are effective for some actions but note
that you have to specify such options \emph{before} the action option. Options
after an action option will be ignored.

The followings are other normal command-line options:

\begin{clopt}{\sopt{w}, \lopt{view}}[default]
Set \ci{mode} to |view|. Texdoc will open the best results it found by the best
application it knows. See Section~\ref{sec:modes}.
\end{clopt}

\begin{clopt}{\sopt{l}, \lopt{list}}
Set \ci{mode} to |list|. Texdoc will show the lists of all relevant documents
it found and ask you which to open. When used with the \lopt{nointeract}
option, Texdoc will just show the lists of documents and exit successfully.
\end{clopt}

\begin{clopt}{\sopt{m}, \lopt{mixed}}
Set \ci{mode} to |mixed|. If Texdoc finds only a document for your keyword,
then it opens it, and if it finds multiple, then it will behave like in the
|list| mode.
\end{clopt}

\begin{clopt}{\sopt{s}, \lopt{showall}}
Set \ci{mode} to |showall|. It also behave like in a |list| mode, but show all
the documents it found including documents which seem to be not a good result.
\end{clopt}

\begin{clopt}{\sopt{i}, \lopt{interact}}[default]
Texdoc may require user reactions, \eg ask you which document to open in the
|list| mode. Internally, this sets \ci{interact\_switch} to |true|.
\end{clopt}

\begin{clopt}{\sopt{I}, \lopt{nointeract}}
If this is specified, Texdoc will never request any input from users.
Internally, this sets \ci{interact\_switch} to |false|.
\end{clopt}

\begin{clopt}{\sopt{M}, \lopt{machine}}
This make the result list of Texdoc machine-readable. Only effective for the
|list| mode and its friends. This option implies \lopt{nointeract}.
Internally, this sets \ci{machine\_switch} to |true|.
\end{clopt}

\begin{clopt}{\sopt{q}, \lopt{quiet}}
This suppress warnings and most error messages. Internally, this sets
\ci{verbosity\_level} to minimum.
\end{clopt}

\begin{clopt}{\sopt{v}, \lopt{verbose}}
Make Texdoc to print additional information such as viewer commands which are
invoked by the program. Internally, this sets \ci{verbosity\_level} to maximum.
\end{clopt}

\begin{clopt}{%
  \code{\sopt{d} \meta{list}}, \code{\lopt{debug}=\meta{list}},
  \sopt{D}, \lopt{debug}}
This sets \ci{debug\_list} to show debugging information in the specified
category. You can specify multiple categories with a comma-separated list. If
you specify |-D| or |--debug| without specifying a list, it activates all
available debug categories.
\end{clopt}

\begin{clopt}{\code{\sopt{c} \meta{name}=\meta{value}}}
This enables you to set arbitrary configuration items |«name»| to |«value»|
from the command-line. See Section~\ref{sec:conf} for the list of all available
configuration items.
\end{clopt}

\subsection{Environment variables}
\label{sec:envvar}

Some environment variables affect the behavior of Texdoc. The environment
variables can be roughly categorized into three groups.

First, as well as other programs in {\TL}, searching path for the Texdoc are
determined with the Kpathsea variables like |TEXMF*|. The mechanism of Kpathsea
is complex, so please refer to `the {\TL} Guide', which supposed to be open by
|texdoc texlive|, and the document of Kpathsea for the details. Usually, Texdoc
searches documents for the |TEXMF| trees including |TEXMFDIST|, |TEXMFLOCAL|,
and |TEXMFHOME| and trees specified in |TEXDOCS|. It also self-locates its own
program using |TEXMF| search path. In addition, Texdoc uses |TEXMFROOT| and
|TEXMFVAR| to look for the tlpdb database and the cache respectively.

Second, the viewers for opening documents can be controlled by some environment
variables. They all correspond to some \ci{viewer\_\meta{ext}}
setting.\footnote{Old names of environment variables, namely
|TEXDOCVIEW_\{html,dvi,md,txt,pdf,ps\}| and
|TEXDOC_VIEWER_\{HTML,DVI,MD,TXT,PDF,PS\}|, are deprecated but still work.} You
can append |_texdoc| to every name in the first column: this wins over every
other name. These variables can be split by colon |:| and the first non-nil
occurrence is used. If a viewer command contains colon, please specify it by
\ci{viewer\_\meta{ext}}.
%
\begin{center}
\begin{tabular}{lll}
Variable & Use for & Configuration item \\ \hline
|BROWSER| & HTML files & |viewer_html| \\
|DVIVIEWER| & DVI files & |viewer_dvi| \\
|MDVIEWER| & Markdown files & |viewer_md| \\
|PAGER| & Text files & |viewer_txt| \\
|PDFVIEWER| & PDF files & |viewer_pdf| \\
|PSVIEWER| & PS files & |viewer_ps| \\
\end{tabular}
\end{center}

Third, on Unix systems, locale-related variables such as |LANG| and |LC_ALL|
are used for the default value of \ci{lang}.

\subsection{Precedence of configuration sources}
\label{sec:prec}

Values for a particular setting can come from several sources. The sources are
looked at in the following order and the first value found is always used:
%
\begin{enumerate}
\item Command-line options.
\item Environment variables ending with |_texdoc|.
\item Other environment variables.
\item Values from configuration files (see below).
\item Hard-coded defaults that may depend on the current machine.
\end{enumerate}

The configuration files are found in the directories \path{TEXMF/texdoc}, where
\path{TEXMF} is the kpathsea variable, in the order given by this variable.
Inside each directory, three files are recognized, in this order:
%
\begin{enumerate}
\item |texdoc-«platform».cnf| where |«platform»| is the name of the current
  platform (defined as the name of the directories where the {\TL}
  binaries are located, for example |x86_64-linux|). This may be useful when
  an installation is shared across machines with different architectures
  needing different settings, for example for viewers. Their use is not
  recommended in any other situation.
\item |texdoc.cnf| is the recommended file for normal use.
\item |texdoc-dist.cnf| is useful for installing a newer version of texdoc
  (including its default configuration file) in your home while retaining
  the use of the previous file for your personal setting; see
  {\TexdocRepo} for instructions on running the development version.
\end{enumerate}

\subsection{Exit codes}
\label{sec:exit}

The status of Texdoc, which is whether or not it was successfully executed and
an expected result was delivered, can be grasped by seeing its exit code. This
will be useful to manage Texdoc by other programs. The current exit codes of
Texdoc are:
%
\begin{description}[left=2em]
\item[\code{0}] Success.
\item[\code{1}] Internal error.
\item[\code{2}] Usage error.
\item[\code{3}] No documentation found.
\end{description}

\section{Customizing the Search Results}

\subsection{An overview of how Texdoc works}

When you type |texdoc «keyword»|, first filenames related to the |«keyword»|
are collected from the two following sources.
%
\begin{enumerate}
\item the |TEXMF| trees containing documentation that specified by the
  \href{https://www.tug.org/kpathsea/}{kpathsea} variable |TEXDOCS|: Texdoc
  collects all files containing |«keyword»| in their filenames or in the names
  of their parent directories.
\item the {\TL} Database (|texlive.tlpdb|): Texdoc searches for packages named
  |«keyword»| or containing a file |«keyword».«ext»| where |«ext»| may be |sty|
  or |cls|, and collects all document files of the found packages.
\end{enumerate}

Second, the collected filenames are filtered by their extensions. After
filtering, only files with known extensions (listed in \ci{ext\_list}) remain.
If Texdoc cannot find any documentation, the fuzzy search will find the closest
package name to the |«keyword»| and it will recollect document files (see
Section~\ref{sec:fuzzy}).

Third, the collected filenames are given numeric scores by using some
heuristics. For example, a filename |«keyword».pdf| is good, and thus gets a
high score. A name |«keyword»-«lang».pdf| is also good and gets a higher score
if the |«lang»| is the preferred language code in your configuration. In
addition, there are several rules such as ``a name |«keyword»-doc| always wins
over any names |«keyword»whatever|'' and ``files under a directory named
exactly |«keyword»| get bonuses''. Scores may also be adjusted based on the
extensions and some known names or subwords. For example, by default,
|Makefile|s get very low scores because basically they are not
documents.\footnote{Nevertheless, they often end up in the |doc| trees,
because in {\TL}, sources of documents are usually in the same directory as the
documents themselves. Other source files for documents are discriminated by
their extensions.}

Finally, a file with the highest score is opened with a viewer, or the list
of results is shown, depending on the current mode. Usually, only results with
positive scores are displayed, except in the |showall| mode. Results with very
bad scores, that are lower than |-100|, are never displayed.

This model for searching and scoring is efficient, but unfortunately, it is not
perfect: Texdoc may sometimes need a hint, either to find a relevant file or,
more likely, to recognize which of the found files is the best. For example,
suppose you are looking for a document of the {\LaTeX} package |shortvrb|.
Texdoc will successfully find |shortvrb.sty| in the {\TL} package |latex|.
However, Texdoc will not be able to sort a number of documents in the |latex|
package correctly because none of them contains the string |shortvrb|.

For such cases, \emph{aliases} are useful: in the default configuration file,
|shortvrb| is aliased to |base/doc|, so that Texdoc primarily look for
|base/doc| when you type |texdoc shortvrb|. Note that Texdoc will also look for
the original keyword, and that a name can be aliased to more than one new
name.

\subsection{Aliases}
\label{sec:alias}

Aliases can be set in configuration files of Texdoc with the following syntax:
%
\begin{htcode}
alias «original keyword» = «name»
alias(«score») «original keyword» = «name»
\end{htcode}
%
A number of aliases are already set in the default configuration file. You can
also define your own aliases in your personal configuration files (see
Section~\ref{sec:quick-file} or \ref{sec:prec} to know where is that). For example,
you can add a line
%
\begin{htcode}
alias td = texdoc
\end{htcode}
%
to the file to alias |td| to |texdoc|. Then, files in the doc trees matching
exactly with |texdoc|, \ie this document |texdoc.pdf|, will be added to the
result list when you search for the keyword |td|.

By default, files match with the aliased name get a score of |10|. This score
is greater than any result of heuristic scoring. It means that results found
via aliases will always be ranked higher than results found for the original
keyword. If you want the results associated with a particular alias to have a
custom score instead of the default score |10|, you can use the optional
argument to the |alias| directive. This can be useful if you associate multiple
aliases to a keyword and want to specify the priorities for them.

In addition, aliases for |«keyword»-«lang»|, where |«lang»| is your preferred
language's 2-letter code (as detected or configured, see the |lang| option), are
also used when searching for |«keyword»|. Files match with these language-based
aliases get additional bonus scores so that they win over the results for
language-independent aliases.

For concrete examples, please have a look at the default configuration file,
which can be found in the last line of the output for |texdoc -f|. If you think
aliases you defined locally should be added to the default configuration,
please share the idea to the {\TexdocML} or {\TexdocRepo}.

Aliases are additive: if you define your own aliases for a keyword in your
configuration file, and there are also aliases for the same keyword in the
default configuration, they will add up. To prevent the default aliases from
being applied for a particular keyword, add a line |stopalias «keyword»| in
your personal configuration file. It will preserve the aliases defined before
this directive if any, but prevent all further aliasing on this keyword.

Here are some notes for aliases. First, they are case-insensitive. Second, they
do not work recursively: only aliases set to the original keyword are used.
Third, results found for aliases always get the score defined by the |alias|
directive (|10| by default). The adjustments described in the other subsections
are not applied to results for aliases.

\subsection{Score adjustments}
\label{sec:score}

Scores can be arbitrary adjusted by using the |adjscore| directive in the
configuration files. The adjustment can be done either for global, \ie for any
keywords, or a particular keyword. The syntax is as follow:
%
\begin{htcode}
adjscore «pattern» = «score adjustment»
adjscore(«keyword») «pattern» = «score adjustment»
\end{htcode}
%
When the optional argument |«keyword»| is specified, the adjustment will be
applied only when a user searches for the |«keyword»|. Otherwise, the
adjustment applied always for the |«pattern»|.

Here are a few examples from the default configuration file.

\begin{htcode}
adjscore /Makefile = -1000
adjscore /tex-virtual-academy-pl/ = -50
adjscore(tex) texdoc = -10
\end{htcode}
%
Herein, all files named |Makefile|, and those with suffixes, \eg |Makefile.in|,
will be eliminated from results: by adjusting their score with such a large
negative value, their final score will be less than |-100|. Which means they
will never be displayed. Files from the |tex-virtual-academy-pl| directory, on
the other hand, are not eliminated but get a negative adjustment, since they
are a common source of ``fake'' matches which hide better results. The third
line gives a minus adjustment for results containing |texdoc| only when the
searched keyword is |tex|. Otherwise, such results would get high scores
because the heuristic scoring would assume |texdoc| is the name of
documentation for {\TeX} itself. The adjustment value |-10| is enough to ensure
that those results will have a negative score, so the files containing |texdoc|
will not be displayed unless |showall| mode is used.

Note that the values of scores, such as the default score for aliases and the
range of heuristic scores, may be changed in a future version of Texdoc.
Therefore, you may need to adapt your score adjustments after updating Texdoc
in the future.

\subsection{Extensions and basenames of files}
\label{sec:ext}

File extensions regarded as documents by Texdoc can be specified with the
configuration item \ci{ext\_list}. By default, files with extensions |pdf|,
|html|, |htm|, |txt|, |dat|, |md|, |ps|, and |dvi| and files without extension
are recognized as documents.

During the scoring process, the configuration item \ci{badext\_list} is also used:
files with a ``bad'' extension appearing in this list will get a lesser score.

In the practical filenames, things which are not actually extensions can follow
dots, \eg |readme.fr| and |readme.texlive|. Thus, in addition to the lists of
known extensions, Texdoc has lists for known basenames: \ci{basename\_list} and
\ci{badbasename\_list}. These configuration items are respectively similar to
\ci{ext\_list} and \ci{badext\_list}. In short, a file will be selected if
either its extension or its basename is known, and get a lesser score if either
is known to be ``bad''.

\subsection{Common filenames}
\label{sec:common-names}

Document filenames of packages are often in the form of |«package»-doc|.
Respecting this culture in the {\TeX} community, Texdoc gives a special score
for files named |«package»«suffix»|, where |«suffix»| is an element of the list
\ci{suffix\_list}. Please refer to the shipped configuration file for the
default setting. Also, feel free to suggest additional elements for the list to
the {\TexdocML} with a real-life example.

\subsection{Fuzzy search}
\label{sec:fuzzy}

When the normal search cannot find any document in {\TL}, Texdoc will execute a
fuzzy search. The fuzzy search tries to find the closest name of a package in
{\TL}\footnote{Note that the feature searches only package names at this point.
Other objects such as aliases cannot be found by the fuzzy search.} to the
input |«keyword»|. The results of the fuzzy search are shown in an info
message, which can be seen by using the command-line option |-v|.

The default allowance of Levenshtein distance is |3|. You can change this
allowance by using the configuration item \ci{fuzzy\_level}. Results of fuzzy
search could be different among executions if multiple package names have the
same Levenshtein distance to the input.

\section{Configuration items}
\label{sec:conf}

Configuration files are line-oriented text files. Comments begin with a |#|
and run to the end of line. Lines containing only space are ignored. Space at
the beginning or end of a line, as well as around an |=| sign, is ignored.
Apart from comments and empty lines, each line must be of one of the following
forms.
%
\begin{htcode}
«configuration item» = «value»
alias «original keyword» = «name»
alias(«score») «original keyword» = «name»
stopalias «original keyword»
adjscore «pattern» = «score adjustment»
adjscore(«keyword») «pattern» = «score adjustment»
\end{htcode}

We will concentrate on the |«configuration item»| part here, since other
directives have already been presented (Section~\ref{sec:alias} and
\ref{sec:score}).

In the above, |«value»|  never needs to be quoted: quotes would be interpreted
as part of the value, not as quotation marks (this also holds for the other
directives).

Lines which do not obey these rules raise a warning, as well as unrecognised
values of |«configuration item»|. The |«value»| can be an arbitrary string,
except when the name of the |«configuration item»| ends with:
%
\begin{enumerate}
\item |_list|, then |«value»| is a coma-separated list of strings. Spaces
  around commas are ignored. Two consecutive comas or a coma at the beginning
  or end of the list means the empty string at the corresponding place.
\item |_switch|, then |«value»| must be either |true| or |false|
  (lowercase).
\item |_level| and |_lines|, then |«value»| is an integer.
\end{enumerate}
%
In these cases, an improper |«value»| will raise a warning too.

\begin{confitem}{mode}
  {\choice{view, list, mixed, showall}}[default: \code{view}]
Set the  mode to the given value. The various modes have been described
in Section~\ref{sec:modes}.
\end{confitem}

\begin{confitem}{interact\_switch}{\choice{true, false}}[default: \code{true}]
Turn on or off interaction. Turning interaction off prevents Texdoc from asking
you to choose a file to view when there are multiple choices, so it just prints
the list of files found.
\end{confitem}

\begin{confitem}{suffix\_list}{\meta{list}}[default: empty]
Set the list of known suffixes to |«list»| (see
Section~\ref{sec:common-names}). Default is the empty list, but see the
shipped configuration file for more.
\end{confitem}

\begin{confitem}{ext\_list}
  {\meta{list}}[default: \code{pdf, html, htm, txt, dat, md, dvi, ps,}]
Set the list of recognised extensions to |«list»|. This list is used to filter
and  sort the results that have the same score (with the default value: pdf
first, etc). Two special values are recognised:
%
\begin{itemize}
\item \emph{The empty element}. This means files without extensions, or more
  precisely without a dot in their name. This is meant for files like
  |README|, etc. The file is assumed to be plain text for viewing purpose.
\item |*| means any extension. Of course if it is present in the list, it
  can be the only element!
\end{itemize}

There is a very special case: if the searched |«name»| has |.sty| extension,
Texdoc enters a special search mode for |.sty| files (not located in the same
place as real documentation files) for this |«name»|, independently of the
current value of |ext_list| and |mode|. In an ideal world, this wouldn't be
necessary since every sty file would have a proper documentation in pdf, html
or plain text, but\dots

For each |«ext»| in |ext_list| there should be a corresponding |viewer_«ext»|
value set. Defaults are defined corresponding to the default |ext_list|, but
you can add values if you want. For example, if you want Texdoc to be able
to find man pages and display them with the |man| command, you can use
%
\begin{htcode}
ext_list = pdf, html, htm, 1, 5, txt, md, dvi, ps,
viewer_1 = man
viewer_5 = man
\end{htcode}

As a special case, if the extension is |sty|, then the |txt| viewer is used;
similarly, if it is |htm| the |html| viewer is used. Otherwise, the |txt|
viewer is used and a warning is issued.
\end{confitem}

\begin{confitem}{badext\_list}{\meta{list}}[default: \code{txt, dat,}]
Set the list of ``bad'' extensions to |«list»|. Files with those extensions get
a malus of |1| on their heuristic score if it was previously positive.
\end{confitem}

\begin{confitem}{basename\_list}
  {\meta{list}}[default: \code{readme, 00readme}]
Set the list of ``known'' basenames to |«list»|. Files with those basenames
are selected regardless of their extension. If the extension is unknown, the
text viewer will be used to view the file.
\end{confitem}

\begin{confitem}{badbasename\_list}
  {\meta{list}}[default: \code{readme, 00readme}]

Set the list of ``bad'' basenames to |«list»|. Files with those names get a
malus of |1| on their heuristic score if it was previously positive.
\end{confitem}

\begin{confitem}{viewer\_\meta{ext}}{\meta{command}}
Set the viewer command for files with extension |«ext»| to |«command»|. For
files without an extension, |viewer_txt| is used. Note that there is no
|viewer_| variable. In |«command»|, |%s| can be used as a placeholder for the
filename, which is otherwise inserted at the end of the command. The command
can be an arbitrary shell construct.
\end{confitem}

\begin{confitem}{lang}{\meta{2-letter code}}
Set your preferred language. By default, it reflects your system's locale.
\end{confitem}

\begin{confitem}{verbosity\_level}{\meta{number}}[default: \code{2}]
Set the verbosity level to |«number»|. At level~|3|, errors, warnings and
informational messages will be printed on stderr; |2| means only errors and
warnings, |1| only errors and |0| nothing except internal errors (obviously not
recommended).
\end{confitem}

\begin{confitem}{debug\_list}{\meta{list}}[default: empty]
Set the list of activated debug categories. Available debug categories are
|config|, |files|, |search|, |score|, |texdocs|, |tlpdb|, |version|, |view|,
and |all| to activate all of these. Implies |--verbose|. Debug information are
printed on standard error.
\end{confitem}

\begin{confitem}{max\_lines}{\meta{number}}[default: \code{20}]
Set the maximum number of results to be printed without confirmation in list,
mixed or showall mode. This setting has no effect if interaction is disabled.
\end{confitem}

\begin{confitem}{machine\_switch}{\choice{true, false}}[default: \code{false}]
Turn on or off machine-readable output. With this option active, the value of
|interact_switch| is forced to |false|, and each line of output is
%
\begin{htcode}
«argument»\metatab«score»\metatab«filename»
\end{htcode}
%
where |«argument»| is the name of the argument to which the results correspond
(mainly useful if there were many arguments), {\metatab} is the tab (ASCII \#9)
character, and the other entries are pretty self-explanatory. Nothing else is
printed on stdout, except if an internal error occurs (in which case exit code
will be |1|). In the future, more tab-separated fields may be added at the end
of the line, but the first 3 fields will remain unchanged.

Currently, there are two additional fields: a two-letter language code, and an
unstructured description, both taken from the CTAN catalogue (via the {\TL}
database). These fields may be empty, and they are not guaranteed to keep the
same meaning in future versions of Texdoc.
\end{confitem}

\begin{confitem}{zipext\_list}{\meta{list}}[default: empty]
List of supported extensions for zipped files. Allows compressed files with
names like |foobar.«zipext»|, where |«zipext»| is an element in the given
|«list»|, to be found and unzipped before the viewer is started (the temporary
file will be destroyed right after).

Warning: Support for zipped documentation is not meant to work on windows, a
Unix shell is assumed! If you add anything to this list, please make sure that
you also set the corresponding \ci{unzip\_\meta{zipext}} item for each
|«zipext»| in the list. At the same time, make sure you are using blocking
(i.e., not returning immediately) viewers.

Remark: {\TL} does not ship any compressed documentation file, so this option
is mainly useful with re-packaged versions of {\TL} that do, for example one in
Linux distributions.
\end{confitem}

\begin{confitem}{unzip\_\meta{zipext}}{\meta{command}}[no default]
The unzipping command for compressed files with extension |«zipext»|. You
should set one for each item in the \ci{zipext\_list}. The command must print
the result on stdout, like |gzip -d -c| does.
\end{confitem}

\begin{confitem}{rm\_file}{\meta{command}}[default: \code{rm -f}]
Commands for removing files on your system. Only useful for zipped documents
(see \ci{zipext\_list}).
\end{confitem}

\begin{confitem}{rm\_dir}{\meta{command}}[default: \code{rmdir}]
Commands for removing directories on your system. Also only useful for zipped
documents (see \ci{zipext\_list}).
\end{confitem}

\begin{confitem}{lastfile\_switch}{\choice{true, false}}[default: \code{false}]
If set to true, prevents Texdoc from reading any other configuration file after
this one (they will be reported as ``disabled'' by |texdoc -f|). Mainly useful
for installing a newer version of Texdoc in your home and preventing the
default configuration file from older versions to be used (see the
\href{https://github.com/TeX-Live/texdoc}{README} for instructions on how to do
so).
\end{confitem}

\begin{confitem}{fuzzy\_level}{\meta{number}}[default: \code{3}]
Set the allowance of Levenshtein distance to |«number»| for fuzzy search. At
level~|0|, the fuzzy search feature is disabled.
\end{confitem}

\begin{confitem}{texlive\_tlpdb}{\meta{path}}[no default]
This enables you to specify the |«path»| for the database |texlive.tlpdb| which
Texdoc uses for searching documents. By default, Texdoc uses
\path{TEXMFROOT/tlpkg/texlive.tlpdb}. Usually, you don't have to (or should
not) set this configuration item. This is only useful when you want to use
Texdoc within a {\TL}-based {\TeX} distribution which does not ship
|texlive.tlpdb|.
\end{confitem}

\section{Licence}
\label{sec:licence}

The current version of Texdoc program and its documentation are copyright
2008--2021 Manuel Pégourié-Gonnard, Takuto Asakura, the {\TL} Team.

They are free software: you can redistribute it and/or modify it under the
terms of the GNU General Public License as published by the Free Software
Foundation, either version 3 of the License, or (at your option) any later
version.

This program is distributed in the hope that it will be useful, but
\emph{without any warranty}; without even the implied warranty of
\emph{merchantability} or \emph{fitness for a particular purpose}. See the
GNU General Public License for more details.

You should have received a copy of the GNU General Public License along with
this program. If not, see \url{http://www.gnu.org/licenses/}.

\bigskip

Previous work (Texdoc program) in the public domain:
%
\begin{itemize}
\item Contributions from Reinhard Kotucha (2008).
\item First texlua versions by Frank K\"uster (2007).
\item Original shell script version by Thomas Esser, David Aspinall, and Simon
  Wilkinson.
\end{itemize}

\bigskip

\begin{center}
\Large\bfseries
Happy {\TeX}ing!
\end{center}

\end{document}
% vim: set ambiwidth=single spell:
